\subsection{图卷积神经网络}
卷积神经网络(CNN)无法处理非欧几里得(Non Euclidean Structure)的数据,学术上的表达是传统的离散卷积在Non Euclidean Structure的数据上无法保持平移不变性。也就是说在拓扑图中每个顶点的相邻顶点数目都可能不同,那么也就无法用一个同样的尺寸的卷积核来进行卷积运算。CNN无法处理Non Euclidean Structure的数据,又希望在拓扑图上有效的提取空间特征来进行学习,所以GCN成为研究重点。
图结构的数据在现实应用中非常常见,比如: Social Networks, Citation Netwoks, Knowledge Graphs等。图半监督学习的设定是,在给定的图结构的数据中,只有少部分节点是有标记的,大部分节点是未标记的。其任务就是预测出未标记节点的label。

经典的方法大概可以分为两类:Standard Approach和Embedding-based Approach。来自阿姆斯特丹大学的作者Thomas N.Kipf Max Welling采用图卷积神经网络来进行半监督分类\cite{kipf2016semi}。接下来,将对图卷积神经网络进行简要的介绍。

图卷积神经网络(Graph Convolutional Networks),对于一个图$G=(V,E)$,我们有输入$X$是一个$N\times D$的矩阵,表示每个节点的特征,同时有图的邻接矩阵$A$。我们希望得到一个$N\times F$的特征矩阵$Z$,表示我们学得的每个节点的特征表示,其中$F$是我们希望得到的维度。

对于$L$层的神经网络来说,可以表示为$H^{l+1}=f(H^l,A)$,其中$H^0=X,H^L=Z$。

传统的图像上的卷积,可以看作多余每一个节点,都加上其邻居节点的信息。而对于图来说,也可以通过$f(H^l,A)=\delta(AH^lW^l)$完成类似的操作。从公式中不难看出,乘以邻接矩阵$A$就相当于对于每个节点,都加上其邻居节点的信息,这也可以看作是传统卷积的一种拓展。值得注意的是,这样的操作并没有加上自身的信息,除非这个节点有自环。因此,为了保证每个节点都能加上自己的信息,都强行加上自环。即:$\widetilde{A}=A+I$。与此同时,为了计算的简便,需要对A矩阵进行正则化操作,即:$D^{-\frac{1}{2}}AD^{\frac{1}{2}}$。

经过对于每个节点加上自环和正则化后,传播规则变为
\[f(H^l,A)=\delta(\widehat{D}^{-\frac{1}{2}}\widehat{A}\widehat{D}^{\frac{1}{2}}H^lW^l),\quad \widehat{A}=A+I
\]

一个信号$x$与一个滤波器$g\theta^{'}$的卷积定义为
\[
g\theta^{'}*x \approx \sum_{k=0}^{K}\theta_k^{'}T_k(\widehat{L})x,
\]

其中,$\widehat{L}=\frac{2}{\lambda_{max}}L-I_{N}$,$T_k(x)=2xT_{k-1}(x)-T_{k-2}(x),T_{0}(x)=1,T_{1}(x)=x$

在线性图卷积网络中,我们进一步的约束$\lambda_{max}\approx2$,在这个约束下,图卷积被简化为下式
\[
g\theta^{'}*x \approx 
\theta_0^{'}x+\theta_1^{'}(L-I_N)x=\theta_0^{'}x+\theta_1^{'}D^{-\frac{1}{2}}AD^{\frac{1}{2}}
\]