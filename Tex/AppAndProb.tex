\section{应用}
图表示学习和GNN具有跨不同任务和领域的许多应用。尽管可以由GNN的每个类别直接处理常规任务,包括节点分类、图分类、网络嵌入、图生成和时空图预测,但其他与图相关的常规任务,例如节点聚类、链接预测、图分区也可以由GNN解决。我们也可以从具体领域来看一看GNN的应用。

\subsection{计算机视觉}
 GNN在计算机视觉中的应用包括场景图生成、点云分类与分割、动作识别等。

识别对象之间的语义关系有助于理解视觉场景背后的含义。场景图生成模型的目的是将图像解析为一个由对象及其语义关系组成的语义图。另一个应用是通过生成给定场景图的真实图像来反转该过程。由于自然语言可以被解析为每个词代表一个对象的语义图,因此在给定文本描述的情况下合成图像是一个很有前途的解决方案。

识别视频中包含的人类行为有助于从机器的角度更好地理解视频内容。一些解决方案检测视频片段中人体关节的位置。由骨骼连接的人体关节自然形成一个图形。给定人类关节位置的时间序列,可以应用STGNNs来学习人类的行为模式。

此外,GNN在计算机视觉中的应用方向还在不断增加。包括人机交互、语义分割、视觉推理和问答等。

\subsection{自然语言处理}
GNN在自然语言处理中的一个常见应用是文本分类。GNN可以利用文档或单词之间的相互关系来推断文档标签。

\subsection{交通}
在智能交通系统中,准确预测交通网络中的交通速度、交通量或道路密度至关重要。使用STGNN能够解决流量预测问题,可以把交通网络看作是一个时空图,其中节点是安装在道路上的传感器,边是通过节点对之间的距离来测量的,每个节点都有一个窗口内的平均交通速度作为动态输入特征。另一个工业级应用是出租车需求预测。

\subsection{推荐系统}
基于图的推荐系统将项目和用户作为节点。通过学习项目与项目之间,用户与用户之间,用户与项目之间以及内容信息之间的关系,基于图的推荐系统可以产生高质量的推荐。推荐系统的关键在于给项目对用户的重要性进行评分,因此可以将其转换为链路预测问题。

GNNs的应用不限于上述领域和任务,其在许多问题上的应用已有初步探索,如程序验证、程序推理、社会影响预测、对抗性攻击预防、电子健康记录建模、脑网络、事件检测和组合优化等。

\section{困难与挑战}

虽然GNN已经被证明具有学习图数据的能力,但由于图数据的复杂性,目前仍然存在大量挑战。

一方面,深度学习的成功在于深度神经结构,然而ConvGNN的性能随着图卷积层数的增加而显著下降,因此模型深度受限。其次,GNN的可扩展性是以破坏图的完整性为代价的。无论是使用抽样还是聚类,模型都会丢失部分图形信息。通过采样,一个节点可能会错过其有影响力的邻居;通过聚类,一个图可能被剥夺一个不同的结构模式。如何权衡算法的可扩展性和图的完整性是未来的研究方向。另外,图本质上是动态的,节点或边可能出现或消失,节点/边输入可能会随着时间而改变。为了适应图的动态性,需要新的图卷积。虽然STGNN可以部分地解决图的动态性问题,但在动态空间关系的情况下如何进行图卷积这一领域还较少有人研究。


\section{总结}

在本文中,我们对图表示学习和图神经网络进行了全面的概述。在文章的第一部分,我们介绍了图表示学习的基本概念,将图表示学习方法分为了普通的节点表示学习方法和图神经网络;在第二部分,我们详细介绍了几种较为著名的节点表示学习方法,包括\emph{DEEPWALK}、\emph{LINE}和\emph{Node2vec};接着,我们详细介绍了四个类别的图神经网络:循环图神经网络、卷积图神经网络、图自动编码器和时空图神经网络;最后,我们对图表示学习和图神经网络的实际应用和当前面临的挑战进行了简要介绍。
